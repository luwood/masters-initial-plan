%% Adaptado de 
%% http://www.ctan.org/tex-archive/macros/latex/contrib/IEEEtran/
%% Traduzido para o congresso de IC da USP
%%*****************************************************************************
% N�o modificar

\documentclass[twoside,conference,a4paper]{IEEEtran}

%******************************************************************************
% N�o modificar
\usepackage{IEEEtsup} % Defini��es complementares e modifica��es.
\usepackage[latin1]{inputenc} % Disponibiliza acentos.
\usepackage[english,brazil]{babel}
%% Disponibiliza Ingl�s e Portugu�s do Brasil.
\usepackage{latexsym,amsfonts,amssymb} % Disponibiliza fontes adicionais.
\usepackage{theorem} 
\usepackage[cmex10]{amsmath} % Pacote matem�tico b�sico 
\usepackage{url} 
%\usepackage[portuges,brazil,english]{babel}
\usepackage{graphicx}
\usepackage{amsmath}
\usepackage{amssymb}
\usepackage{color}
\usepackage[pagebackref=true,breaklinks=true,letterpaper=true,colorlinks,bookmarks=false]{hyperref}
\usepackage[tight,footnotesize]{subfigure} 
\usepackage[noadjust]{cite} % Disponibiliza melhorias em cita��es.
%%*****************************************************************************

\begin{document}
\selectlanguage{brazil}
\renewcommand{\IEEEkeywordsname}{Palavras-chave}

%%*****************************************************************************

\urlstyle{tt}
% Indicar o nome do autor e o curso/n�vel (grad-mestrado-doutorado-especial)
\title{Plano de Pesquisa}
\author{%
 \IEEEauthorblockN{Lu�sa Madeira Cardoso}
}

%%*****************************************************************************

\maketitle

%%*****************************************************************************
% Resumo do trabalho
\begin{abstract}
Resumo
\end{abstract}

% Indique tr�s palavras-chave que descrevem o trabalho
\begin{IEEEkeywords}
 Palavras-chave
\end{IEEEkeywords}

%%*****************************************************************************
% Modifique as se��es de acordo com o seu projeto

\section{Introdu��o}
A replica��o � um dos t�picos mais estudados na �rea de sistemas distribu�dos\cite{schneider1990}.

\section{Trabalho Proposto}

Trabalho proposto. 

%******************************************************************************
% Refer�ncias - Definidas no arquivo Relatorio.bib
 +-------------+

\bibliographystyle{IEEEtran}

\bibliography{Plan}


%******************************************************************************

\end{document}
